\documentclass[french]{article}

% Page
    \usepackage[a4paper,%
            left=1in,right=1in,top=1in,bottom=1in]{geometry}
    \usepackage{lscape}

% Abstract
    \usepackage{abstract} % Allows abstract customization
    % Set the "Abstract" text to bold
    \renewcommand{\abstractnamefont}{\normalfont\bfseries}
    % Set the abstract itself to small italic text
    \renewcommand{\abstracttextfont}{\normalfont\small\itshape} 

% Title
    \usepackage{titlesec} % Allows customization of titles
    
% Endnotes
	% Uncomment this line if using endnotes "\endnote{}"
	% \usepackage{endnotes}
    
% Headers from page 2 on
    \usepackage{fancyhdr}
    \pagestyle{fancy}
    \fancyhf{}
    \usepackage{lastpage}

% MACROS
    % Define keywords macro command
    \providecommand{\keywords}[1]{\textbf{\textit{Keywords---}} #1}


% MATH SUPPORT
    % The amssymb package provides various useful mathematical symbols
    \usepackage{amssymb}
    % The amsthm package provides extended theorem environments
    \usepackage{amsthm}
    % The newtxmath package provides additional math symbol support
    	% in Times New Roman symbols, etc.
    \usepackage{newtxmath}

% FONTS
    \usepackage{microtype} % Slightly tweak font spacing for aesthetics
    \usepackage[utf8]{inputenc}
    \usepackage{newtxtext} % Makes default font Adobe Times New Roman
    \usepackage[T1]{fontenc}
    \usepackage{rotating}
    \usepackage{babel}
  
% LINES
	% Spacing
	\usepackage{setspace} % See \doublespacing command at the top of content.tex
    % Numbering
    \usepackage{lineno,xcolor} 	% See \linenumbers at the top of content.tex

% COMMENTS
	\usepackage[colorinlistoftodos]{todonotes} % allows margin comments
    % See examples in content.tex, and here for manual: 
    % http://www.ctan.org/pkg/todonotes
	\usepackage{soul} % allows for highlighting

% APPENDIX
    \usepackage[toc,page,title,titletoc,header]{appendix} % pour les annexes
    \renewcommand{\appendixtocname}{Appendices} % indique le nom de la table des annexes dans la toc
    \renewcommand{\appendixpagename}{Appendices} % Nom du titre de la page des annexes

% CODE
    \usepackage{tcolorbox}
    \usepackage{xcolor}
    \newcommand{\reducedstrut}{\vrule width 0pt height .9\ht\strutbox depth .9\dp\strutbox\relax}
    \usepackage{minted}

% SEQUENCES
    \usepackage{texshade}
    \usepackage{booktabs, colortbl}

% GRAPHICS
    \usepackage{svg}
    \usepackage{graphicx} % More advanced figure inclusion
    \usepackage{float} % For specifying table/figure locations, i.e. [ht!]
    
    % The printlen command allows the user to print the exact text width or height.
    % This is useful, when trying to create graphics (outside of LaTeX, of course)
    % with the optimal dimensions. See here for usage: http://www.ctan.org/pkg/printlen
    \usepackage{printlen}
    \usepackage{tikz}
    \usepackage{caption}

% TABLES
    \usepackage{pgfplotstable}
    \pgfplotstableset{
begin table=\begin{longtable},
end table=\end{longtable},
}
    \pgfplotsset{compat=1.17}
    \usepackage{longtable} % For long tables that span multiple pages
    \newcommand{\sym}[1]{\rlap{#1}}% For symbols like *** in tables
    \usepackage{tabularx} % Allows advanced table features
    \newcolumntype{M}{>{\centering\arraybackslash}m{4.4em}}
    \newcolumntype{L}[1]{>{\raggedright\arraybackslash}p{#1}}
    \newcolumntype{C}[1]{>{\centering\arraybackslash}p{#1}}
    \newcolumntype{R}[1]{>{\raggedleft\arraybackslash}p{#1}}
    \usepackage{relsize} % Allows precise adjustment of font size,
    	%useful for fitting tables to page width
    \usepackage{multirow}
    \usepackage{booktabs}
    

% REFERENCES
    \usepackage{cite}
	\usepackage{hyperref} % For hyperlinks in the PDF
	\hypersetup{
	colorlinks=true,
	linkcolor=black,
	citecolor=black,
    urlcolor=blue,
    }
	\usepackage{chicago}

    
